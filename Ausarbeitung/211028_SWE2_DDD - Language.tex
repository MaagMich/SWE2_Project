\newSec[DDDLang]{Ubiquitous Language}{2}

Die \textit{Ubiquitous Language} ist eine einheitliche Sprache, auf die sich Entwickler und Domänenexperten einigen und gemeinsam verwenden. In diesem Projekt ist aus Zeitgründen keine solche einheitliche Sprache definiert worden, da die Rollen \textit{Domänen-Expert*in} und \textit{Entwickler*in} von der gleichen Person eingenommen wurden. Verständnisprobleme sind hierbei nicht zu erwarten.\footnote{Aber nie ganz auszuschließen.}

\clearpage
Um zukünftigen Anwendenden den Zusammenhang zwischen Domänenmodell und Code aufzuzeigen, soll anschließend eine Zuodnung von Objekten des Domänenmodells zu Klassen im Code folgen.
\begin{table}[!ht]
\begin{tabular}{ll}
Domänenmodell & Klassenbezeichnung \\ \hline
Drohne        & parrotControl      \\
Data Output   & parrotIMU          \\
Control Input & parrotTransmitter  \\
Controller    & PoseController     \\
Position      & Vector3D           \\
Orientation   & Vector3D           \\
User Input    & \ROS-Message         
\end{tabular}
\end{table}

Darüber hinaus kann die Dokumentation der zugrundeliegenden Studienarbeit im weiteren Sinne als \textit{Ubiquitous Language}-Dokument verstanden werden. Hier werden alle verwendeten Klassen beschrieben. \note{Auf das genannte Dokument kann an dieser Stelle noch nicht verwiesen werden, da es noch nicht vollständig existiert. Die Abgabe der Studienarbeit ist auf nach der Abgabe dieser Ausarbeitung datiert.}

Ein separates Dokument, welches ausschließlich eine \textit{Ubiquitous Language} definiert, wäre sinnvoll.








