% Seiten-Layout
\documentclass[a4paper,12pt]{scrartcl}	%Seitenlayout allgemein
%\documentclass[a4paper,12pt]{article}		%Seitenlayout allgemein
\usepackage[left=3cm, right=2.5cm, top=2.5cm, bottom=2.5cm, footskip=0.7cm]{geometry}						%Seitenränder
\usepackage{fancyhdr}				%Kopf- und Fußzeilen-Manipulation
\pagestyle{empty}					%Seitenzahl, empty=ohne, 
\fancyhf{} 							%alle Kopf- und Fußzeilenfelder bereinigen
\usepackage{lscape}					%Querformat in Bereich
%\usepackage[section]{placeins}


% Text-Layout
\usepackage[ngerman]{babel} %deutsche Sprache+Umlaute
\usepackage[latin1, utf8]{inputenc}
\usepackage[T1]{fontenc} %Codieung7-auf8-Bit
\usepackage{marvosym} %Symbole
\DeclareUnicodeCharacter{20AC}{\EUR} %z.B.Euro-Symbol
\usepackage{nameref} %für-Überschriften
\addto\captionsngerman{\renewcommand{\refname}{Literaturverzeichnis}}
\usepackage{lmodern}				%Vektorisierte Schrift
\usepackage[scaled]{uarial}			%Arial
\usepackage[onehalfspacing]{setspace}	%1,5-facher Zeilenabstand



% Überschriften



% Forms-Formatierung
\usepackage{parskip}				%Absätze
\usepackage{array}					%Tabellen
\usepackage[format=hang,font={footnotesize,sc}, 
labelfont={bf},margin=1cm,aboveskip=5pt,
position=top]{caption} 				%Tabellenunterschriften, evtl. Formatieren
\usepackage{pifont}					%Dingbats-Symbole (siehe Tabelle)
\usepackage{enumerate}				%Aufzählungen: a A i I 1
\usepackage{graphicx}				%Bilder

\usepackage{placeins}

% Verzeichnisse
\usepackage{titletoc}				% Inhaltsverzeichnis umdefinieren
\usepackage{tocloft}					% Abbildungsverzeichnis umdefinieren
\usepackage{acronym}				%Abkürzungsverzeichnis



% Codes
\usepackage{xcolor}					% Color
\usepackage{listings}				%Weiteres Verzeichnis
\lstset{captionpos=b}				% Caption after Code



% Anhang
\usepackage[toc,page]{appendix}
\renewcommand{\appendixtocname}{Anhang}
\renewcommand{\appendixpagename}{Anhang}






% Drawings
\usepackage{tikz}				%Graphik-Elemente / Zeichnen



% Programming Functions
\usepackage{amsmath}
\usepackage{amssymb}
\usepackage{empheq}
\usepackage{mathrsfs}
\usepackage{theorem}
\usepackage{xifthen}




% Layout Überschriften im TOC
\newcommand{\TocLayoutE}[3]{\titlecontents{#1}[#2em]{}{\contentslabel{0em}}{\vspace*{#3cm}}{\titlerule*[0.3pc]{.}\contentspage}}
\newcommand{\TocLayoutD}[4]{\titlecontents{#1}[#2em]{}{\contentslabel{#4em}}{\vspace*{#3cm}}{\titlerule*[0.3pc]{.}\contentspage}}
\newcommand{\TocLayoutA}[4]{\titlecontents{#1}[#2em]{}{\contentslabel{#4em}}{\vspace*{#3cm}}{\titlerule*[0.3pc]{}}}

\newcommand{\TocStdSectionE}{\TocLayoutE{section}{0}{0}}
\newcommand{\TocStdSectionD}{\TocLayoutD{section}{1.5}{0}{1.5}}
\newcommand{\TocStdSectionA}{\TocLayoutA{section}{0}{0}{1.5}}

\newcommand{\TocStdSubSectionE}{\TocLayoutE{subsection}{3}{0.25}}
\newcommand{\TocStdSubSectionD}{\TocLayoutD{subsection}{3}{0.25}{2.25}}
\newcommand{\TocStdSubSectionA}{\TocLayoutA{subsection}{3}{0.25}{2.25}}

\newcommand{\TocStdSubSubSectionE}{\TocLayoutE{subsubsection}{4.5}{0.25}}
\newcommand{\TocStdSubSubSectionD}{\TocLayoutD{subsubsection}{4.5}{0.25}{3}}
\newcommand{\TocStdSubSubSectionA}{\TocLayoutA{subsubsection}{4.5}{0.25}{3}}



\usepackage{chngcntr}

% Layout PaygeStyle
\newcommand{\PagestyleStdPre}{\pagestyle{empty}}
\newcommand{\PagestyleStdFormal}{\pagestyle{fancy} \setcounter{page}{1} \pagenumbering{Roman} \fancyhf[fr]{\thepage} \TocStdSectionE \TocStdSubSectionE \TocStdSubSubSectionE \captionsetup[figure]{list=no}}
\newcommand{\PagestyleStdContent}{\pagestyle{fancy} \setcounter{page}{1} \pagenumbering{arabic} \fancyhf[fr]{\thepage} \TocStdSectionD \TocStdSubSectionD \TocStdSubSubSectionD \captionsetup[figure]{list=yes}}
\newcommand{\PagestyleStdSuf}{\pagestyle{empty} \TocStdSectionA \TocStdSubSectionA \TocStdSubSubSectionA \captionsetup[figure]{list=no}}
\newcommand{\PagestyleStdAppendix}{\pagestyle{fancy} \setcounter{page}{1} \pagenumbering{roman} \renewcommand{\thepage}{\thesection-\arabic{page+1}} \counterwithin{page}{section} \TocLayoutA{section}{1.5}{0}{1.5}}

\renewcommand{\headrulewidth}{0pt}
\renewcommand{\footrulewidth}{0pt}



% New Colors
\definecolor{Raspberry}{RGB}{188, 17, 66}
\definecolor{GitStyle}{RGB}{188, 17, 66}
\definecolor{BranchStyle}{RGB}{188, 17, 66}
\definecolor{CommitStyle}{RGB}{188, 17, 66}



\definecolor{CodeStyleBG}{RGB}{210, 210, 210}
\definecolor{CodeStyleNum}{RGB}{0, 0, 0}
\definecolor{CPPStyleKey}{RGB}{43, 151, 192}
\definecolor{CPPStyleMeth}{RGB}{116, 83, 31}
\definecolor{CPPStyleNum}{RGB}{0, 0, 0}
\definecolor{CPPStyleString}{RGB}{193, 21, 21}
\definecolor{CPPStyleComment}{RGB}{0, 128, 0}

% Layout Codes
\lstdefinestyle{Style_Java}{
	language=Java,
	basicstyle=\tiny,
	backgroundcolor=\color{CodeStyleBG},
	commentstyle=\color[rgb]{.17,.55,.17},
	keywordstyle=\color[rgb]{1,.08,.58},
	numberstyle=\tiny\color[rgb]{.1,.2,.3},
	stringstyle=\color[rgb]{0,0,.53},
	numbers=left,
	stepnumber=1,
	breakatwhitespace=true,
	breaklines=true,
	showspaces=false,
	showstringspaces=false,
	showtabs=false,
	tabsize=2
}

\lstdefinestyle{Style_CPP}{
	language=C++,
	basicstyle=\tiny,
	backgroundcolor=\color{CodeStyleBG},
	commentstyle=\color{CPPStyleComment},
	keywordstyle=\color{CPPStyleKey},
	numberstyle=\tiny\color{CPPStyleNum},
	stringstyle=\color{CPPStyleString},
	numbers=left,
	stepnumber=1,
	breakatwhitespace=true,
	breaklines=true,
	showspaces=false,
	showstringspaces=false,
	showtabs=false,
	tabsize=4
}

\lstdefinestyle{Style_Bash}{
	language=bash,
	basicstyle=\tiny,
	backgroundcolor=\color{CodeStyleBG},
	commentstyle=\color[rgb]{.17,.55,.17},
	keywordstyle=\color[rgb]{1,.08,.58},
	numberstyle=\tiny\color[rgb]{.1,.2,.3},
	stringstyle=\color[rgb]{0,0,.53},
	numbers=none,
	stepnumber=1,
	breakatwhitespace=true,
	breaklines=true,
	showspaces=false,
	showstringspaces=false,
	showtabs=false,
	tabsize=2
}


\lstdefinestyle{Style_XML}{
	language=XML,
	basicstyle=\tiny,
	backgroundcolor=\color{CodeStyleBG},
	commentstyle=\color[rgb]{.17,.55,.17},
	keywordstyle=\color[rgb]{1,.08,.58},
	numberstyle=\tiny\color[rgb]{.1,.2,.3},
	stringstyle=\color[rgb]{0,0,.53},
	numbers=none,
	stepnumber=1,
	breakatwhitespace=true,
	breaklines=true,
	showspaces=false,
	showstringspaces=false,
	showtabs=false,
	tabsize=2
}

\lstdefinestyle{Style_IOC}{
	language=XML,
	basicstyle=\tiny,
	backgroundcolor=\color{CodeStyleBG},
	commentstyle=\color[rgb]{.25,.25,.25},
	keywordstyle=\color[rgb]{0,0,0},
	numberstyle=\tiny\color[rgb]{0,0,0},
	stringstyle=\color[rgb]{0,0,0},
	numbers=left,
	stepnumber=1,
	breakatwhitespace=true,
	breaklines=true,
	showspaces=false,
	showstringspaces=false,
	showtabs=false,
	tabsize=2
}


\newcommand{\CodePkg}[1]{Paket \texttt{\textit{\textcolor{Raspberry}{#1}}}}
\newcommand{\CodeClass}[1]{Klasse \texttt{\textcolor{CPPStyleKey}{#1}}}
\newcommand{\CodeMeth}[1]{Methode \texttt{\textcolor{CPPStyleKey}{#1}}}
\newcommand{\CodeStruct}[1]{\texttt{\textcolor{CPPStyleKey}{#1}}}
\newcommand{\CodeVar}[1]{Variable \texttt{\textcolor{CPPStyleKey}{#1}}}

\newcommand{\Branch}[1]{\textcolor{BranchStyle}{Branch } \texttt{#1}}
\newcommand{\Commit}[1]{\textcolor{CommitStyle}{Commit } \texttt{#1}}












