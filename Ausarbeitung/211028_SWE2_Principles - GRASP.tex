\newSec[PrincGRASP]{GRASP}{2}


\newSec[PrincGRASPKopp]{Kopplung}{3}
Der Einsatz von \ROS\ deutet an dieser Stelle auf eine lose Kopplung hin (siehe \refCap{MusterObserver}).\footnote{vgl. Vorlesungsfolien \glq Programming Principles\grq (Seite 50)} 

Durch polymorphe Methodenaufrufe liegt eine mäßige Kopplung vor. Dies wurde in der \CodeClass{ControllerSystem} der \CodeMeth{addController} umgesetzt.
https://github.com/MobMonRob/ROSLabDrohne/blob/8b5d30dbdfbd41a7e10fa8d512db42cad5bb6d16/Code/Controller/include/Controller/ControllerSystem.h\#L28 ff.

Abgesehen von den genannten Punkten ist der zugrundeliegende Code eher stark gekoppelt, da häufig statische Methodenaufrufe sowie Methodendefinitionen verwendet werden.




Brücke ist eher lose gekoppelt. \missing[prüfen!]





\newSec[PrincGRASPKopp]{Kohäsion}{3}

mit niedriger Kohäsion wird das single responsibility principle.
Wird angewandt. Beispiel: Controller-Aufbau in Domain.


