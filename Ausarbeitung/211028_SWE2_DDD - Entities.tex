\newSec[DDDEntities]{Entities}{2}
Nachfolgend werden die \textit{Entites} aus \refImg{fig:DomainMod} beschrieben.


\newSec[DDDEntitiesDrone]{Drone}{4}
Das Objekt \textit{Drone} bildet die physische Drohne ab. Das physisch vorhandene Pendant kann Zustände ändern (ruhend, fliegend, verunfallt), so auch das Objekt der Domäne. 


\newSec[DDDEntitiesData]{Data Output}{4}
In dem Objekt \textit{Data Output} werden Posen aus Beschleunigungsdaten und der Ausrichtung der Drohne erzeugt. Es handelt sich hier explizit nicht um eine \textit{Pure Fabrication} (\textit{Factory}), da hier eine Funktionalität der Drohne (Übergabe von Beschleunigungsdaten und Ermittlung der lokalen Pose) umgesetzt wird. Diese Funktionalität ist für die Domäne essentiell.
Für die Umrechnung von Beschleunigungsdaten zu einer Position werden Integrale gebildet und somit Zustände zur Laufzeit verändert. Dies erhebt das Objekt \textit{Data Output} zu einem \textit{Entity}.


\newSec[DDDEntitiesController]{Controller}{4}
Im \textit{Controller}-Objekt werden entsprechend regelungstechnischer Ansätze Stellgrößen gebildet. Hierzu werden im Allgemeinen \textit{PID-Regelglieder} eingesetzt, diese enthalten ein integralbildendes Regelglied. Hieraus folgt, wie bereits für das Objekt \textit{Data Output} angemerkt, die Eigenschaft als \textit{Entity}.











