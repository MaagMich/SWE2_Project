\newSec[PrincSOLID]{SOLID}{2}


\missing[Sem5 VL4]



\newSec[PrincSOLIDSingle]{Single Responsibility Principle}{3}

\missing[Was genau soll hier beschrieben werden? nochmal Vorlesung anhören!]

\missing[Hier evtl eine Tabelle mit allen Klassen (nach Layer und Alphabet geordnet?) und dem von uns definierten Zweck?]


Beispiel: Klassen aus DroneController\_pkg mit jeweiligen Aufgaben beschreiben.
=> Michael




\newSec[PrincSOLIDOC]{Open/Closed Principle}{3}

\missing[Vorlesung Sem5 VL xx]

Offen für Erweitgerung. Geschlosen gegenüber Veränderungen.

\missing[TODO: VORLESUNG ANHÖREN!]

Beispiel: Value und TimedValue ??

Beispiel: Der ganze Input/Output Kram für die Controller

Beispiel Closed: ControllerSystem

Gegenbeispiel: Integral1 und Integral2, Erklärung: zu Aufwändig und idR nicht benötigt in diesem Anwendungsfall.


\newSec[PrincSOLIDLisk]{Liskov Substitution Principle}{3}

\missing[Vorlesung Pricipals Folie 23.]

Kompletter Code ist theoretisch Kovariant.

Hier als Beispiel das Controller-Array in ControllerSystem.




\newSec[PrincSOLIDInter]{Interface Segregation Principle}{3}

implizit Single responsibility.

Klient soll nur das gezeigt bekommen, was er tatsächlich auch benötigt.

Beispiel: DroneController\_pkg






\newSec[PrincSOLIDDep]{Dependency Inversion Principle}{3}



\missing[Transmitable in PoseController]

Die Brücke.

IMUable => PoseBuilder
PoseControlable => parrotTransmitter


PoseControlable ist zwar maßgeblich virtual, aber implenentiert eine Methode (transmitAction). Damit sind wir uns unklar, ob das noch als dependecy inversion zählt.
Auch so bei DroneControlable => parrotStatus























