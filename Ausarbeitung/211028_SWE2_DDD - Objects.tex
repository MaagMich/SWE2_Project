\newSec[DDDVO]{Value Objects}{2}

Ein beliebiges Objekt ohne eigene Identität wird auch als \textit{Value Object} (\textit{VO}) bezeichnet.
Die Kategorisierung als solches ergibt sich durch die \textit{immutable}-Eigenschaft. Heraus leitet sich eine Gleichheit der Objekte bei gleichen Attribut-Werten ab.

Nachfolgend werden die \textit{Value Objects} aus \refImg{fig:DomainMod} beschrieben.


\newSec[DDDVOPose]{Pose}{4}
Eine Pose ist eine mathematische Beschreibung für eine Punkt und eine Ausrichtung in einem Raum. Nimmt die Drohne eine andere Pose ein, wird hierfür eine separate Instanz generiert.

Die von der Pose gekapselten Objekte \textit{Position} und \textit{Orientierung} werden als mathematische Vektoren im 3D-Raum abgebildet.


\newSec[DDDVOControlInput]{Control Input}{4}
Der \textit{Control Input} ist ein Objekt, welches Steuerungsdaten an die physische Drohne übersendet.
Im Sinne der Domäne entspricht das Objekt \textit{Control Input} einem \textit{Value Object}.
Sämtliche Parameter werden bei der Initialisierung des Objekts übergeben.


\newSec[DDDVOInput]{User Input}{4}
Als \textit{User Input} werden regelungstechnische Führungsgrößen verstanden, welche an den Controller der Drohne übergeben werden. Diese besitzen keinen Zustand.



