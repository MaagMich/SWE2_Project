\newSec[PrincYAGNI]{YAGNI}{2}

\missing[Sem5 VL6-2 Min36]



You ain't gonna need it

Erst Code hinzufügen, wenn benötigt.

Wurde im Projekt teilweise durchgeführt - git-Historie durchschauen... Fuck ist das aufwändig... -.-

Mein allgemeiner Programmier-Stil entspricht nicht vollständig diesem Prinzip.
Grundlegende Funktionalitäten werden hinzugefügt, WEIL sie benötigt werden, nicht WENN es soweit ist :D
Immerhin weiß man ja schon so grob, was alles im Projekt benötigt wird... Zugegeben, hier wird oft erst die leere Funktion eingebaut und der benötigte Code dann, wenn es an den jeweiligen Test/Ausprobieren geht.
Beispiel: Vector3D::rotate() Body

Basis von TDD => Schreibe nur, was du wirklich brauchst.