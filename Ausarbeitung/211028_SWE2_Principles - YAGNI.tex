\newSec[PrincYAGNI]{YAGNI}{2}
\textit{YAGNI} (\textit{You ain't gonna need it}) ist ein Prizip, nach dem Code nur erzeugt wird, wenn dieser unmittelbar benötigt wird. Auch absehbar benötigte Funktionalität ist zurückzustellen.

In diesem Projekt wurde dieses Prinzip aus der Präferenz des Entwicklers heraus nicht angewandt: Grundlegende Funktionalitäten werden hinzugefügt, \textit{WEIL} sie benötigt werden, nicht \textit{WENN} sie benötigt werden.\\
An dieser Stelle unterstellen wir eine gute Projektplanung, welche einen Einsatz des YAGNI-Prinzips quasi obsolet macht.

Abweichend hiervon wurde dieses Prinzip im Zusammenhang mit der Anwendung von TDD umgesetzt (siehe \refCap{TDD}).


