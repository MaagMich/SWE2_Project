\newSec[Refa]{Refactoring}{1}






\newSec[RefaSmell]{Code Smells}{2}


\newSec[RefaSmellDup]{Duplicated Code}{3}
Wie bereits in \refCap{PrincDRY} angemerkt, findet sich Code-Duplikate im Code:

\zB\\
\Commit{edd7726743b9eb67e159112919224d54c54be670}\\
\CodeClass{FixedPoint<T>} (mehrfach in verschiedenen Methoden)\\
\zB\\
\CodeMeth{operator+(const FixedPoint<TIn>\& FP) [const]} (mutable und const Spezifizierer)\\
Identischer Code ist in nachfolgend genannten Methoden vorhanden.\\
\href{https://github.com/MobMonRob/ROSLabDrohne/blob/73f3400f4866b27200a37be5d3a80fa8394b2c48/Code/Domain/include/Domain/FixedPoint.h\#L145}{Link: Domain/FixedPoint.h\#L145}\\
\href{https://github.com/MobMonRob/ROSLabDrohne/blob/73f3400f4866b27200a37be5d3a80fa8394b2c48/Code/Domain/include/Domain/FixedPoint.h\#L157}{Link: Domain/FixedPoint.h\#L157}


\newSec[RefaSmellLong]{Long Method}{3}
\href{https://github.com/MobMonRob/ROSLabDrohne/blob/73f3400f4866b27200a37be5d3a80fa8394b2c48/Code/PosControl/src/PoseBuilder.cpp\#L68}{Link: PosControl/PoseBuilder.cpp\#L68}
\Commit{73f3400f4866b27200a37be5d3a80fa8394b2c48}\\
\CodeClass{PoseBuilder}, \CodeMeth{updatePose}

Wie innerhalb der Methode ersichtlich ist wird ein manipulierender Code durch einen Block eingefasst. Dieser Block kann in eine eigene Methode verschoben werden und somit in der betrachteten Methode zu einer Funtionalität umgeändert werden.




\newSec[RefaSmellLarge]{Large Class}{3}
\href{https://github.com/MobMonRob/ROSLabDrohne/blob/73f3400f4866b27200a37be5d3a80fa8394b2c48/Code/Domain/include/Domain/FixedPoint.h}{Link: Domain/FixedPoint.h}
\Commit{73f3400f4866b27200a37be5d3a80fa8394b2c48}\\
\CodeClass{FixedPoint<T>}

Um die Funktionalität (Arithmetik) gewährleisten zu können, sind in dieser (aktuell) Klasse 35 Methoden definiert.
Nach unserer subjektiven Definition ist eine Klasse ab 15 Methoden eine \textit{Large Class}.


\newSec[RefaSmellComments]{Code Comments}{3}
An deiser Stelle sollen lediglich Links aufgezählt werden.
\begin{itemize}
\item \href{https://github.com/MobMonRob/ROSLabDrohne/blob/d39df8957ec8659b8a6683d8d8ba853db01a82ff/Code/Domain/src/TimedValue.cpp\#L17}{Link: Domain/TimedValue.cpp\#L17}\\
ToDo-Kommentar, nicht in den Code!

\item \href{https://github.com/MobMonRob/ROSLabDrohne/blob/d39df8957ec8659b8a6683d8d8ba853db01a82ff/Code/Domain/src/TimedValue.cpp\#L39}{Link: Domain/TimedValue.cpp\#L39}\\
ToDo-Kommentar, nicht in den Code!

\item \href{https://github.com/MobMonRob/ROSLabDrohne/blob/d39df8957ec8659b8a6683d8d8ba853db01a82ff/Code/PosControl/src/PoseBuilder.cpp\#L103}{Link: PosControl/PoseBuilder.cpp\#L103}\\
impliziter ToDo-Kommentar, nicht in den Code!
\end{itemize}

Informative Comments und damit sinnvoll:
\begin{itemize}
\item \href{https://github.com/MobMonRob/ROSLabDrohne/blob/d39df8957ec8659b8a6683d8d8ba853db01a82ff/Code/Domain/src/Vector3D.cpp\#L99}{Link: Domain/Vector3D.cpp\#L99}
\item \href{https://github.com/MobMonRob/ROSLabDrohne/blob/d39df8957ec8659b8a6683d8d8ba853db01a82ff/Code/DroneController/src/Rotor.cpp\#L26}{Link: DroneController/Rotor.cpp\#L26}
\end{itemize}


\newSec[RefaSmellSwitch]{Switch Statement}{3}
\href{https://github.com/MobMonRob/ROSLabDrohne/blob/3c5aeed610150bd19f75241af48a2777cb4c582a/Code/PosControl/src/MainClass.cpp\#L67}{Link: PosControl/MainClass.cpp\#L67}
\Commit{3c5aeed610150bd19f75241af48a2777cb4c582a}\\
\CodeClass{MainClass} \CodeMeth{callbackKeys}

Grundsätzlich sollen \textit{Switch Statements} vermieden werden.
In diesem Fall wird davon ausgegangen, dass keine weitere Funktionalität zur \CodeClass{parrotControl} hinzugefügt wird.





\newSec[RefaRefa]{Refactoring}{2}

\newSec[RefaRefaExtract]{Method Extraction}{3}
Angewandt auf die in \refCap{RefaSmellLong} beschriebene Methode.\\
\Commit{9dd6821a9c63bf1682806b8ab319d6004247d3ca}\\
\CodeClass{PoseBuilder} \CodeMeth{updatePose(IMUState State)}

Hier waren semantisch gekapselte Aufgaben innerhalb einer Methode vorhanden. Diese wurden in separate Methode ausgelagert und nachfolgend der Aufruf der Use Cases.\\
vorher:\\
\href{https://github.com/MobMonRob/ROSLabDrohne/blob/ea7d3d0640e8cde5156c6604e2432fa71061914b/Code/PosControl/src/PoseBuilder.cpp\#L71}{Link: PosControl/PoseBuilder.cpp\#L71}

nacherher:\\
\href{https://github.com/MobMonRob/ROSLabDrohne/blob/9dd6821a9c63bf1682806b8ab319d6004247d3ca/Code/PosControl/src/PoseBuilder.cpp\#L71}{Link: PosControl/PoseBuilder.cpp\#L71}\\
\href{https://github.com/MobMonRob/ROSLabDrohne/blob/9dd6821a9c63bf1682806b8ab319d6004247d3ca/Code/PosControl/src/PoseBuilder.cpp\#L94}{Link: PosControl/PoseBuilder.cpp\#L94}\\
\href{https://github.com/MobMonRob/ROSLabDrohne/blob/9dd6821a9c63bf1682806b8ab319d6004247d3ca/Code/PosControl/src/PoseBuilder.cpp\#L108}{Link: PosControl/PoseBuilder.cpp\#L108}


\newSec[RefaRefaRename]{Rename Class}{3}
In der Vorlesung wird das Umbenennen einer Methode angesprochen. Ziel ist hier, die Verständlichkeit des Codes zu erhöhen.
In dem dieser \Arbeit\ zugrundeliegenden Code wurden Klassen gefunden, bei welchen die Bezeichnung nicht selbstsprechend erscheint. Um dieser Problematik entgegenzuwirken, soll eine dieser betreffenden Klassen umbenannt werden. Hierbei handelt es sich um die \CodeClass{PosBridge} nach  \CodeClass{MainClass}.\\
\Commit{3c5aeed610150bd19f75241af48a2777cb4c582a}\\
\href{https://github.com/MobMonRob/ROSLabDrohne/blob/3c5aeed610150bd19f75241af48a2777cb4c582a/Code/PosControl/src/MainClass.cpp}{Link: PosControl/MainClass.cpp}




\newSec[RefaRefaDeleg]{Interface Extraction}{3}
Da sich die Hardware (Drohnen-PlugIn) im Projektverlauf geändert hat, wurde diese Chance genutzt, eine Bridge einzuführen. Dies hatte den Hintergrund, einen Umstieg auf andere Hardware zukünfitig zu erleichtern.
Hierzu sind diverse, \tw\ virtuelle\footnotemark[1], Klassen entstanden.

\Commit{2b1bccb0f9b2671ace7fb91b6ed86e2be2284596}























