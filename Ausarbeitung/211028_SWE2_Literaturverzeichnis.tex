\clearpage
\addcontentsline{toc}{section}{Literaturverzeichnis}
\newcounter{IndexLiteratur}
\newcommand{\NewBibItem}[2]{\stepcounter{IndexLiteratur} \bibitem[\theIndexLiteratur]{#1} #2,\newline}

% Layout Commands
\newcommand{\Published}[1]{veröffentlicht #1}
\newcommand{\Modified}[1]{verändert #1}
\newcommand{\Requested}[1]{abgefragt #1}
\newcommand{\Online}[1]{online,  #1\newline}
\newcommand{\Version}[1]{#1. Auflage}
\newcommand{\ISBN}[1]{ISBN #1}
\newcommand{\Article}[1]{Artikel \textit{#1}}
\newcommand{\Pages}[1]{Seite #1}
\newcommand{\NA}{-unbekannt-}



\begin{thebibliography}{\theIndexLiteratur}

%Bücher



% Einleitung






%ROS
\NewBibItem{ROS}{ROS - Robot Operating System}
\Online{http://ros.org}
\Published{\NA}, \Requested{10.04.2022}


\NewBibItem{ROSObserer1}{Understanding ROS Topics}
\Online{http://wiki.ros.org/ROS/Tutorials/UnderstandingTopics}
\Published{\NA}, \Modified{18.07.2019}, \Requested{27.03.2022}



%Testing
\NewBibItem{gmockFAQ}{Legacy gMock FAQ}
\Online{https://google.github.io/googletest/gmock\_faq.html}
\Published{\NA}, \Requested{22.04.2022}










\end{thebibliography}

\Anmerkung{Wird hier ein Veröffentlichungsdatum als \grqq -unbekannt-\grqq\ markiert, so konnte diese Angabe weder auf der entsprechenden Webseite, noch in deren Quelltext ausfindig gemacht werden.}

\clearpage