\newSec[PrincDRY]{DRY}{2}
Dieses Prinzip fördert das Schonen von Ressourcen. Daher wurde dieses Prinzip in großen Teilen des zugrundeliegenden Code angewandt.\\
Sichtbar ist dies beondersbei Vererbung, da bereits vorhandene Funktionalität aus der Basisklasse übernommen wird und keine Neuimplementierung dieser stattfindet.
\zB\\
\CodeClass{TimedValue } erbt von \CodeClass{Value} und \CodeClass{Timestamp}, wobei nur wenige notwendige Funktionen (Konstruktoren und einfache Arithmetik) hinzugefügt werden.
\href{https://github.com/MobMonRob/ROSLabDrohne/blob/fe35195d45d7246eacdac0165036204d47e10ed6/Code/Domain/include/Domain/TimedValue.h}{Link: Domain/TimedValue.h}\\
\href{https://github.com/MobMonRob/ROSLabDrohne/blob/fe35195d45d7246eacdac0165036204d47e10ed6/Code/Domain/include/Domain/Value.h}{Link: Domain/Value.h}\\
\href{https://github.com/MobMonRob/ROSLabDrohne/blob/fe35195d45d7246eacdac0165036204d47e10ed6/Code/Domain/include/Domain/Timestamp.h}{Link: Domain/Timestamp.h}


Negativ Beispiel siehe \refCap{RefaSmellDup}.


\newSec[PrincDRYProof]{Nachweis}{4}
Um die getroffene positiv-Aussage zu bestätigen, sollte an dieser Stelle ein Nachweis mit einem geeigneten \textit{Duplication Checker}-Tool eingebracht werden.
Leider war es den Autor:innen nicht möglich, dies mit geeignetem Erfolg umzusetzen. Wir bitten an dieser Stelle um Anerkennung der Bemühungen.

\begin{itemize}
\item http://www.ccfinder.net/ccfinderxos.html (nicht ausführbar)
\item https://pmd.github.io/latest/pmd\_userdocs\_cpd.html (zu aufwändig)
\item https://github.com/Acumatica/AntiPlagiarism (leider nur c\#)
\end{itemize}

