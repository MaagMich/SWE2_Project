\newSec[Test]{Software Tests}{1}



Es gibt folgende Test-Arten:
\begin{itemize}
\item Unit Test
\item Integration Test
\item System Test
\item Acceptance Test
\item \missing[fehlt hier was?]
\end{itemize}


Allerdings wird in dieser Ausarbeitung nur auf Unit Tests eingegangen.









\newSec[TestUnit]{Unit Tests}{2}



Wahl fällt auf \textit{banditcpp}, da es sich hierbei um ein \textit{headers only} Framework handelt.
Daraus folgt, dass die Test-Umgebung nicht in den Produktiv-Code eingebunden werden muss. Es kann quasi als PlugIn \glqq aufgesetzt\grqq\ werden.

\missing[Bandit bietet leider keine Code Coverage...]


\newSec[TestUnitUsed]{Umgesetzte Unit Tests}{3}

verweis auf eine tst.cpp Datei (oder mehrere?)


\newSec[TestUnitMock]{Mocks}{4}

Test callTransmitter Test\_Application/test.cpp



\newSec[TestUnitATRIP]{ATRIP-Regeln}{3}

wurde angewandt. Verweis auf einen beliebigen Test.





\newSec[TestCover]{Code Coverage}{3}


\missing[Reka fragen]


using \texttt{coverlet} with \texttt{Visual Studio}



\newSec[TestCoverStart]{Getting Started}{4}
\begin{itemize}
\item installiere Microsoft.NET.Sdk
\item installiere Microsoft.NET.Test.Sdk (NuGet.org)



\end{itemize}


using \texttt{Google Test} with \texttt{Visual Studio}





\newSec[TDD]{TDD}{2}

\textit{Test Driven Development} (\textit{TDD}) ist ein Prozess in der Software-Entwicklung, wonach jede Klasse \bzw\ Funktion im produktiven Code bereits im Vorfeld durch einen Test abgedeckt wird.

Hierbeit unterscheidet sich das \textit{TDD} von \textit{Test First} dadurch, dass \textit{TDD} lediglich jeweils einen Test vor dem produktiven Code liegt, wobei \textit{Test First} eine beliebige Anzahl an Tests bereitstellen kann, bevor produktiver Code entsteht. \missing[Quelle die letzte Vorlesung - Sem6 VL 3 oder so?]

Im zugrundeliegenden \texttt{git}-Repository wird die Entwicklung einer Klasse nach dem \textit{Test First}-Prinzip im \Branch{createFixedPoint} mit dem \Commit{a4f9bd383c32a7d6d1b0be2319764a664199c2d2} gestartet.

